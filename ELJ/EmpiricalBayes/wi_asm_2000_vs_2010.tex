\documentclass[preprint,12pt]{article}

\usepackage{algorithmic}
\usepackage{algorithm}
\usepackage{enumerate}
\usepackage{enumitem}
\usepackage{graphics}
\usepackage{graphicx}
\usepackage{geometry}
\usepackage{amsmath}
\usepackage{wrapfig}
\usepackage{subfig}
\usepackage{framed}
\usepackage{color}
\usepackage{soul}
\usepackage{bm}

\usepackage{multirow}
\usepackage[T1]{fontenc}
\usepackage[latin9]{inputenc}
%\usepackage{units}
\usepackage{esint}

\geometry{legalpaper,  margin=1in}

\newcommand{\CM}[2][green]{ {\sethlcolor{#1} \hl{#2}} }
\newcommand{\KB}[2][cyan]{ {\sethlcolor{#1} \hl{#2}} }

%\makeatother

%\usepackage{babel}fs

\begin{document}
\title{Extracting the components of Specific Partisan Asymmetry for Wisconsin Assembly 2010 census cycle districts}
\author{Kevin Baas and Colin McAuliffe}
\maketitle

\section{Method}

To compare how much of the observed partisan asymmetry in 2010-cycle Wisconsin State Assembly districsts was the result of changes in district designs versus changes in voter sentiment, we varied the first while holding the second constant.

For this analysis we used official districting plans for the 2000 and 2010 census cycles, and actual vote counts from the 2006, 2008, 2010, 2012, 2014, and 2016 assembly elections, at precinct-level resolution, imputing uncontested elections with partisan swing matched presidential election resutls when available, and partisan swing matched federal congress election results when not.
By "partisan swing matched" we mean that the Republican vote counts in all districts are multiplied by a constant factor so that the total vote count matches that for the assembly elections, for the subset of districts in which both elections were contested, and the same is done for the Democratic vote counts.
Election data for 2006, 2008, and 2010 was cross-aggregated at census block resolution to the 2010-cycle districts, and election data for 2012, 2014, and 2016 was cross-aggregated at census block resolution to the 2000-cycle districts. 
 
Since the same exact elections are used in both the 2000 districts analysis and 2010 districts analysis, \emph{all} differences in the results are the consequnece of the redistricting scheme used, and conversely, changes in voter sentiment have exactly \emph{zero} contribution to the difference.

Furthermore, if the 2000 census cycle districts show a partisan asymmetry much lower than the 2010 cycle, then we know that natural "political geography" -- the tendency of Democratic voters to be clustered in cities -- was \emph{not} a significant factor in the partisan asymmetry inherent in 2010 cycle Wisconsin Assembly districts, since any contributions from political geography would be present in \emph{both} districting schemes.

...

\section{Conclusions}
From these comparisons, it is painfully obvious that the specific partisan asymmetry present in the 2010 cycle distritcts is \emph{not} present in the 2000 cycle districts, even when using vote count data from the exact same elections.  Therefore:
\begin{itemize}
\item Since the same exact voter sentiments were used in both analyses, this difference in partisian asymmetry simply cannot be a consequence of changes in voter sentiment.  Since the only thing that we changed was the districting scheme, all observed differences in partisan asymemtry are purely the result of changing the districting scheme.
\item Natural political geography was \emph{not} a significant factor in the partisan asymmetry inherent in Wisconsin Assembly districts in the 2010 cycle.  Were this a significant factor, the exact same effects would be present in the 2000 districts, but they are not.  This finding is consistent with Professor Jowei Chen's findings in his paper "The Impact of Political Geography on Wisconsin Redistricting: An Analysis of Wisconsin's Act 43 Assembly Districting Plan".
\item Simply reverting back to the 2000 census cycle districts would remedy most of the present injustice, as measured in seats affected or ballots affected.
\end{itemize}



%\bibliographystyle{plainnat}
\bibliographystyle{unsrt}
\bibliography{gerrymandering}
\clearpage



\end{document}
